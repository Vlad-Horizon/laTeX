\newcommand{\3}{
\section{3 Створення макету та модулів TeX}
% 
% 
% 
\subsection{3.1 Створення власних модулів}
\Tab{Створити власні модулі можна як в main файлі, так і створити окречий файл, після чого його імпортувати якщо він має використовуватись.
  Усі файли які створюються вами як модулі повині мати розширення .tex.} \par
Зверніть увагу, що імпортувати можна тіки 1-н раз інакше буде помилка, тому імпортуйте всі модулі в main файл або створіть додатковий файл config в якому будуть зберігатись усі імпорти. \par
Для створення модуля необхідно прописати команду \quotesA{\textbackslash{newcommand\{\textbackslash{назва вашої команди}\}\{Тут вміст модуля\}}}. \par
Зверніть увагу, що дві команди не можуть мати однакове ім'я та ваші команди або модулі не повині мати таку ж назву як встановленя або базові. \par
% 
\subsection{3.2 Створення першого модуля}
\Tab{В даному підрозділі буде показано як створити власний модуль (команду) який буде приймати один аргумент.
  Завданням даного модуля буде встановлення жирного текста. Дану операцію можна виконати використовуючи команду \quotesA{\textbackslash{bfseries}},
  але для зручності даний можудь буде викликатись командою \quotesA{\textbackslash{B}}} \par

Перш за все створюється папка для зручного зберігання та навігації, наприклад \quotesA{src>modules>text>B.tex} де \quotesA{B.tex} це назва файлу.
після цього виконується підключення до головного файлу або файлу який містить всі підключення за допомогою команди \quotesA{\textbackslash{input\{шлях до файлу який підключаємо\}}},
та в документі викликається сам модуль \par

Коли виконали підключення переходемо в створений файл в який вписується, що він має робити.
В даному прикладі вміст файлу має такий вигляд [3.1]. \par
\newpage
\Img[Рисунок 3.1 – Вміст файлу B.tex]{./src/images/3.1.png}

В виклий команди необхідно додати текст для коректної роботи модуля та, якщо все виконано правильно, можна побачити результат в файлі main.pdf. \par
% 
\subsection{3.3 Висновок}
\Tab{За допомогою власних модулів можна полегшити роботу з TeX а також зробити свої шаблони для швидкого написання частин документу які постійно повторюються.
  Також розбиття документу на окремі файли допоможе легко орієнтуватись в проекті та працювати з тою частиною документу яка необхідна.} \par
\newpage
}