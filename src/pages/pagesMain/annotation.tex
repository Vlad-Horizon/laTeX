\newcommand{\annotation}{
  \HeadA{\MakeUppercase{Анотація}}
  Даний реферат присвячений основам верстки документів за допомогою системи TeX.
  TeX є мовою та програмним забезпеченням для створення високоякісних наукових і
  технічних документів. В роботі розглядаються основні поняття і принципи верстки,
  такі як структурування документа, використання різних типів шрифтів,
  форматування розділів, таблиць та зображень. \par

  Також розглядаються основні команди та оточення,
  що використовуються в TeX для створення документів.
  Викладаються приклади використання цих команд та детально
  описуються їх функціональні можливості. \par

  Окрема увага приділяється створенню шаблонного файлу, який
  може бути використаний в подальшому для написання інших робіт.
  В шаблонному файлі визначено основні налаштування документа,
  структуру заголовків, нумерацію сторінок, розміщення зображень
  та таблиць, а також включення посилань та бібліографії. \par

  Результати роботи показують, що використання системи
  TeX дозволяє створювати професійно виглядаючі документи з
  високою якістю верстки. Використання шаблонного файлу спрощує
  процес написання і форматування документів, що збільшує
  продуктивність та забезпечує єдність стилю. \par

  % Функціональні
  \thispagestyle{empty}
  \newpage
}