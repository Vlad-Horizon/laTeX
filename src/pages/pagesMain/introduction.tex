\newcommand{\introduction} {
  \section{ВСТУП}
  \Tab{У сучасному інформаційному суспільстві передбачається, що дослідники та
    професіонали здатні чітко та ефективно презентувати свої наукові результати.
    Для досягнення цієї мети можна використовувати різноманітні
    інструменти та технології, але одним з найефективніших і поширених засобів є система
    верстки TeX.} \par

  TeX - це мова та програмне забезпечення, розроблені видатним комп'ютерним вченим
  Дональдом Кнутом. Вона стала популярною серед науковців, математиків, інженерів та
  інших фахівців, оскільки дозволяє створювати високоякісні документи з професійним
  оформленням. Використання TeX дозволяє зосередитися на суті дослідження та аналізу,
  а не на технічних деталях верстки. \par

  Метою даного реферату є ознайомлення з основними принципами верстки документів
  з використанням системи TeX. Робота також включає в себе розробку шаблонного файлу,
  який можна використовувати для створення інших наукових робіт. \par

  В рефераті будуть розглянуті основні принципи верстки документів, зокрема:
  структура документа, включення зображень та таблиць, форматування тексту, використання
  спеціальних символів та формул, посилання та бібліографічні посилання. Крім того,
  будуть надані приклади та практичні поради щодо роботи з системою TeX. \par

  Ознайомлення з основами верстки документів з використанням TeX дозволить
  забезпечити професійне оформлення наукових досліджень та інших документів. Використання
  шаблонного файлу, розробленого у даній роботі, спростить процес створення нових робіт
  та зосередить увагу на змісті. \par

  Таким чином, дана робота пропонує огляд основ верстки документів з використанням
  системи TeX та надає практичні рекомендації щодо створення шаблонного файлу для подальшого
  використання. \par
  \newpage
}