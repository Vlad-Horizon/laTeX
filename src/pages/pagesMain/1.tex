\newcommand{\1}{
  \section{1 Основи верстки документів з використанням TeX}
  % 
  % 
  % 
  \subsection{1.1 Загальний огляд системи TeX}
  \Tab{TeX - це високоефективна система верстки, розроблена Дональдом Кнутом,
    яка знайшла широке застосування серед науковців та фахівців з різних галузей.
    Вона базується на концепції розмітки документів, що дозволяє точно
    контролювати вигляд і структуру тексту. TeX дозволяє створювати професійні
    документи з високою якістю оформлення, включаючи математичні формули,
    графіки, таблиці та багато іншого.} \par

  \HeadC{Структура документа} \par
  У системі TeX документ складається з послідовності команд та тексту.
  Основними структурними елементами документа є розділи, підрозділи,
  параграфи та заголовки. Кожен елемент має свою властиву команду, яка
  визначає його тип та стиль оформлення. \par

  \HeadC{Форматування тексту} \par
  TeX надає багато можливостей для форматування
  тексту, включаючи налаштування шрифтів, вирівнювання,
  інтервали між рядками та керування розмірами сторінки.
  Завдяки цьому, документи можуть мати професійний вигляд і
  відповідати вимогам наукових журналів та конференцій. \par

  \HeadC{Вставка зображень та таблиць} \par
  TeX дозволяє вставляти зображення різних форматів, таких як JPEG, PNG або EPS.
  Зображення можуть бути вирівняні, масштабовані та розміщені в потрібному місці в документі.
  Також можна вставляти таблиці з використанням спеціальних команд, які
  контролюють стиль та розміщення таблиці. \par

  \HeadC{Використання спеціальних символів та формул} \par
  TeX має вбудовану підтримку спеціальних символів, таких як математичні символи, грецькі літери,
  символи розділових знаків тощо. Він також забезпечує потужні інструменти для написання
  математичних формул, що дозволяє створювати складні математичні вирази та рівняння. \par

  \HeadC{Посилання та бібліографічні посилання} \par
  TeX надає зручні засоби для створення посилань в документах, включаючи внутрішні
  та зовнішні посилання. Використання бібліографічних баз даних і стилей оформлення
  дозволяє автоматизувати процес створення бібліографічних посилань та списків літератури. \par

  \HeadC{Розробка шаблонного файлу для TeX} \par
  Для полегшення процесу створення нових документів у системі TeX можна розробити шаблонний файл.
  Шаблонний файл містить вже встановлені команди та налаштування оформлення, які можуть бути використані
  для нових проектів. Це дозволяє зберегти час та забезпечити єдність стилю усіх документів. \par
  % 
  \subsection{1.2 Висновок}
  \Tab{Завершуючи перший розділ реферату, слід зауважити, що оволодіння основами верстки документів з використанням
    системи TeX є незамінним навичкою для всіх, хто прагне створювати професійні та науково-орієнтовані документи.
    Для демонстрації шаблонів та прикладу використання їх, вихідний код даного реферату розміщений на gitHub [1].
  } \par

  % \begin{tabular}{|c|c|c|}
  %   \hline
  %   Заголовок 1         & Заголовок 2         & Заголовок 3         \\
  %   \hline
  %   Рядок 1, стовпець 1 & Рядок 1, стовпець 2 & Рядок 1, стовпець 3 \\
  %   \hline
  %   Рядок 2, стовпець 1 & Рядок 2, стовпець 2 & Рядок 2, стовпець 3 \\
  %   \hline
  % \end{tabular}

  \newpage
}