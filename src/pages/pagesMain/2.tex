\newcommand{\2}{
\section{2 Оформлення документів у системі TeX}
% 
% 
% 
\subsection{2.1 Встановлення та налаштування TeX-системи}
\Tab{Перед початком роботи з TeX необхідно встановити та налаштувати TeX-систему
на своєму комп'ютері.
Для компіляції документів та використання TeX необхідно встановити MikTeX [2] та Strawberry-perl [3].} \par

Після встановлення додатків, необхідно обрати простір для написання TeX документів.
Один з можливих варіантів написання TeX документів є використання Visual Studia Code [4]
в поєднанні з розширенням LaTeX Workshop [5].
До переваг даного способу можна віднести можливість використання контролю версій git яка є за замовчуванням в даному редакторі коду,
можливість розширювати функціонал до рівня IDE (Integrated Development Environment) та можливість налаштування зовнішнього вигляду редактору під себе. \par

Додаткові пакети встановлюються в консолі TeX. Для того, щоб відкрити її відкрийте пуск, знайдіть папку MikTeX, та відкрийте консоль.
Після того як консоль відкриється, перейдіть в вкладку Packages після чого встановіть необхідні модулі. \par
% 
\subsection{2.3 Форматування тексту в TeX}
\Tab{Для форматування тексту в системі TeX використовуються команди які можуть змінювати як якусь частиу тексту
  так і впливати на нього глобально, тобто використовуватись у всьому документі.} \par

Є можливість використовувати усі базові фотмати тексту: \par

\begin{itemize}[itemsep=0cm,labelsep=-1cm,parsep=0pt,topsep=0pt,partopsep=0pt,leftmargin=0pt]
  \item[--] \hspace{1.25cm}{звичайний текст;}
  \item[--] \hspace{1.25cm}{курсивний текст \quotesA{\textbackslash{emph}}, {\emph{приклад}};}
  \item[--] \hspace{1.25cm}{жирний текст \quotesA{\textbackslash{bfseries}}, {\bfseries{приклад}};}
  \item[--] \hspace{1.25cm}{розташування тексту праворуч; \quotesA{\textbackslash{begin\{flushright\}} ваш текст \textbackslash{end\{flushright\}}}}
  \item[--] \hspace{1.25cm}{розташування тексту ліворуч;}
  \item[--] \hspace{1.25cm}{розташування тексту по центру \quotesA{\textbackslash{begin\{center\}} ваш текст \textbackslash{end\{center\}}};}
  \item[--] \hspace{1.25cm}{відступи;}
  \item[--] \hspace{1.25cm}{абзаци; \quotesA{\textbackslash{hspace\{1.25cm\}}}}
  \item[--] \hspace{1.25cm}{заголовки.}
\end{itemize} \par

Також є можливість встановлювати різні шрифти, кольори для тексту й багато інших. \par

Для того аби виконати певну команді, імпортувати пакет, створити власний пакет або
створити наново, спочатку ставиться символ \quotesA{ \textbackslash{} } (зворотній слеш),
після цього вводиться команда яку необхідно виконати.
Також є команди яким необхідна закриваюти \quotesA{ \textbackslash{end\{команда яку закривають\}} },
її необіхдно писати якщо дана комадна була оголошена в документі раніше \quotesA{ \textbackslash{begin\{початок\}}} \par
% 
\subsection{2.3 Початок роботи}
\Tab{Для того аби почати працювати з TeX документами необхідно переконатись що все встановлено та працює.
  Для цього створить папку в якій буде зберігатись ваш проект, в даній папці створіть головний файл \quotesA{main.tex} та в ньому пропишіть базову структуру TeX документу.
} \par
На початку документу викликається команда \quotesA{documentclass} в якій вказується тип документа та додаткові параметри такі як розмір шрифтів, формат сторінки й інші.
Основні параметри вказуються через фігурні дужки, додаткові через квадратні. \par
Після оголошення \quotesA{documentclass} вказується додаткові пакети які будуть використовуватись для написання документу.
Також вказуються ваші модулі які підключаються для їх використання через команду \quotesA{input}, та додаткові налаштування для пакетів. \par
Весь вміст документу прописується в блоці \quotesA{document}. даний блок необхідно закривати як це показано в підрозділі 2.2. \par
Якщо компіляція проекту пройшла успішно і в вас створився файл \quotesA{main.pdf} та при відкритті файлу в вас відображається те саме що ви написали, то це свідчить про те, що TeX працює коректно. \par
% 
\subsection{2.4 Висновок}
\Tab{У даному розділі були розглянуті основні аспекти оформлення документів у системі TeX.
  Було описано процес встановлення та налаштування TeX-системи, а також створення базового шаблону} \par

% \begin{tabular}{|c|c|c|}
%   \hline
%   Заголовок 1         & Заголовок 2         & Заголовок 3         \\
%   \hline
%   Рядок 1, стовпець 1 & Рядок 1, стовпець 2 & Рядок 1, стовпець 3 \\
%   \hline
%   Рядок 2, стовпець 1 & Рядок 2, стовпець 2 & Рядок 2, стовпець 3 \\
%   \hline
% \end{tabular}

\newpage
}